\section{Charge carrier density}
In the following section the determination of the carrier density $n$ is described using three different methods.
All determined carrier densities are listed in tab.\,\ref{tab:ns}.
\subsection{Charge carrier density extrapolation via classical Hall effect}
In the classical Hall effect, the charge carrier density $n_\text{slope}$ is given by 
\begin{align}
    n_\text{slope} = \frac{B}{\rho_{xy}e} = \frac{1}{e \cdot m_\text{slope}}.
    \label{eq:chargeCarrierClassicalHall}
\end{align}
with the slope $m_\text{slope}$ of the Hall resistance $\rho_{xy}$ as the regression parameter.
In our experiment, one expect to see the classical Hall effect at low magnetic fields.
As seen in fig.\,\ref{fig:KlitzingBeispielBild} the slope of the Hall resistance shows a linear behavior for small magnetic fields.
The carrier density is calculated by determining the slope $m_\text{slope}$ of the linear region using a linear regression for the data points for $B<1\,\text{T}$.
The error is calculated by Gaussian error propagation of the error of $m_\text{slope}$ obtained from the linear regression.

\subsection{Charge carrier density via Hall plateaus}
In the quantum Hall regime the carrier density can be calculated from the Hall plateaus with the corresponding filling factors $\nu$:
\begin{align} 
    n_\nu = \frac{\nu B_\nu e}{h} \label{eq:nnu} 
\end{align} 
where $B_\nu$ is the mean magnetic field at the plateau for the corresponding filling factor.
$B_\nu$ is determined by manually reading the center of each plateau, with the error of $B_\nu$ being the uncertainty of the reading. 
For higher filling factors, the plateaus become narrower and the error of the reading becomes slightly smaller, as the midpoint of the plateaus is easier to determine.
The error of $n_\nu$ is then calculated by Gaussian error propagation.

\subsection{Charge carrier density via Shoubnikow-de Haas effect}
\begin{figure}[h]
    \centering
    \includegraphics[width=0.45\textwidth]{../Images/FourierMitGausFür14K.png}
    \caption{Fourier transform of the longitudinal resistivity 
    $\mathcal{F}\left[ \rho^\prime_\text{xx}\right]\left(f_\text B\right)$ 
    for $1.4\,\text{K}$ to determine the frequency of Shoubnikow-de Haas oscillations.\\
    The peak is determined with a fit of a gaussian curve.
    }
    \label{fig:Fourier}
\end{figure}
The Shoubnikow-de Haas effect describes the oscillations of the longitudinal resistivity $\rho_{xx}$ with increasing magnetic field $B$.
It can be explained with the equally spaced Landau levels moving up in energy, when $B$ increases.
The charge carrier density $n_\text{SdH}$ is calculated via
\begin{align}
    n_\text{SdH} = \frac{e}{h\Delta}
    \label{eq:chargeCarrierFourier}
\end{align} 
with the periodicity of oscillations $\Delta\left(\frac{1}{B}\right)$.
To obtain the periodicity $\Delta$, the longitudinal resistivity $\rho^\prime_\text{xx}\left(B\right)=\rho_\text{xx}\left(\frac{1}{B}\right)$ is considered.
Since the measured data points are equally spaced in $B$ instead of $\frac{1}{B}$, an interpolation is performed.
A discrete Fast Fourier Transform is applied to accurately determine the period of the oscillations.
A clear peak is visible.
To accurately locate the peak, a Gaussian is fit to the region and the peak value of the fit is used as $\Delta$.
The error of the peak is estimated to be half the width of the Gaussian $\sigma / 2$.
The Fourier transformed resistivity $\mathcal{F}\left[ \rho^\prime_\text{xx}\right]\left(f_\text B\right)$ for $1.4\,\text{K}K$ is shown in fig. \ref{fig:Fourier}.
The charge carrier densities calculated with eq. \ref{eq:chargeCarrierFourier} are shown in tab. \ref{tab:ns}.
\begin{table}[h]
    \centering
    \begin{tabular}{c|c|c}
        \hline\hline
        Method & $T / \text{K}$ & $n\,/\,10^{15}\text{m}^{-2}$ \\\hline\hline
        slope & 4.2 & $\csname nEins4.2K\endcsname$ \\
        & 3 & $\csname nEins3K\endcsname$ \\
        & 2.1 & $\csname nEins2.1K\endcsname$ \\
        & 1.4 & $\csname nEins1.4K\endcsname $\\\hline
        $\nu$ & 4.2 & $\csname nZwei4.2K\endcsname$ \\
        & 3 & $\csname nZwei3K\endcsname$ \\
        & 2.1 & $\csname nZwei2.1K\endcsname$ \\
        & 1.4 & $\csname nZwei1.4K\endcsname$ \\\hline
        SdH & 4.2 & $\csname FourierN4.2K\endcsname$ \\
        & 3 & $\csname FourierN3K\endcsname$ \\
        & 2.1 & $\csname FourierN2.1K\endcsname$ \\
        & 1.4 & $\csname FourierN1.4K\endcsname$ \\\hline\hline
    \end{tabular}
    \caption{Data of the charge carrier density $n$ calculated with different methods.
    The values are given in units of $10^{15}\,\text{m}^{-2}$. For all following sections
    the method $n_\text{slope}$ is chosen for the determination of $n$. \label{tab:ns}}
\end{table}



