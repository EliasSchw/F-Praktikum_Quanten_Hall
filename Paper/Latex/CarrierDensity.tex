\section{Charge carrier density}
In the following section, the determination of the charge carrier density $n$ is described in three different ways.
All determined charge carrier densities are shown in tab.\,\ref{tab:ns}.
\subsection{Charge carrier density extrapolation from the classical Hall effect}
In the classical Hall effect, the charge carrier density $n_\text{slope}$ is given by 
\begin{align}
    n_\text{slope} = \frac{B}{\rho_{xy}e} = \frac{1}{e \cdot m_\text{slope}}.
    \label{eq:chargeCarrierClassicalHall}
\end{align}
In our experiment, the classical Hall effect is expected to be seen at low magnetic fields.
Like seen in fig.\,\ref{} the slope of the Hall resistivity shows a linear behaviour for small magnetic fields.
The charge carrier density is calculated by determining the slope $m_\text{slope}$ of the linear regime using a linear regression
for the datapoints for $B<1\,\text{T}$.
The error is calculated by gaussian error propagation of the error of $m_\text{slope}$ obtained from the linear regression.

\subsection{Charge carrier density via Hall plateaus}
In the quantum Hall regime, the charge carrier density can be calculated using the Hall plateaus with the corresponding filling factors $\nu$.
Therefore $n_\nu$ can be calculated using eq.\,\ref{eq:nnu}
\begin{align}
    n_\nu = \frac{\nu B_\nu e}{h}
\end{align}
with $B_\nu$ being the mean magnetic field at the plateau for the corresponding filling factor.
$B_\nu$ is determined by manually reading off the midpoint of each plateau, with the error of $B_\nu$ being
the uncertainty of the reading. For higher filling factors, the plateaus are narrower and the error of the reading is getting slightly smaller.
The error of $n_\nu$ is then calculated by gaussian error propagation.

\subsection{Charge carrier density via Shoubnikow-de Haas Effect}
The Shoubnikow-de Haas Effect describes the oscillations of the longitudinal resistivity $\rho_{xx}$ with increasing magnetic field $B$.
It can be explained with the equally spaced Landau levels moving up in energy, when $B$ increases.
The charge carrier density $n_\text{SdH}$ is calculated via
\begin{align}
    n_\text{SdH} = \frac{e}{h\Delta}
    \label{eq:chargeCarrierFourier}
\end{align} 
with the periodicity of oscillations $\Delta\left(\frac{1}{B}\right)$.
To obtain the periodicity $\Delta$, the longitudinal resistivity $\rho^\prime_\text{xx}\left(B\right)=\rho_\text{xx}\left(\frac{1}{B}\right)$ is considered.
Since the measured datapoints are eqally spaced in $B$ instead of $\frac{1}{B}$, an interpolation is performed.
To precisely determine the periodidity of the oscillations, a discrete fast fourier transform is applied.
A clear peak is visible, matching the read off estimated periodicity.
To precisely find the peak, a gaussian is fitted onto the region and the peak value of the fit is used as $\Delta$.
The error of the peak is estimated to be half the width of the gaussian.
The Fourier transformed resistivity $\mathcal{F}\left[ \rho^\prime_\text{xx}\right]\left(f_\text B\right)$ for 
$1.4K$ is depicted in \ref{fig:Fourier}.
\begin{figure}[h]
    \centering
    \includegraphics[width=0.45\textwidth]{../Images/FourierMitGausFür14K.png}
    \caption{Fourier transform of the longitudinal resistivity 
    $\mathcal{F}\left[ \rho^\prime_\text{xx}\right]\left(f_\text B\right)$ 
    for $1.4K$ to determine the frequency of Shoubnikow-de Haas oscillations.
    The peak is determined with a fit of a gaussian curve.
    }
    \label{fig:Fourier}
\end{figure}
The charge carrier densities calculated with \ref{eq:chargeCarrierFourier} are shown in tab. \ref{tab:n_sdh}.
\begin{table}[h]
    \centering
    \begin{tabular}{c|c|c}
        \hline\hline
        Method & $T / \text{K}$ & $n\,/\,10^{15}\text{m}^{-2}$ \\\hline\hline
        slope & 4.2 & $\csname nEins4.2K\endcsname$ \\
        & 3 & $\csname nEins3K\endcsname$ \\
        & 2.1 & $\csname nEins2.1K\endcsname$ \\
        & 1.4 & $\csname nEins1.4K\endcsname $\\\hline
        $\nu$ & 4.2 & $\csname nZwei4.2K\endcsname$ \\
        & 3 & $\csname nZwei3K\endcsname$ \\
        & 2.1 & $\csname nZwei2.1K\endcsname$ \\
        & 1.4 & $\csname nZwei1.4K\endcsname$ \\\hline
        SdH & 4.2 & $\csname FourierN4.2K\endcsname$ \\
        & 3 & $\csname FourierN3K\endcsname$ \\
        & 2.1 & $\csname FourierN2.1K\endcsname$ \\
        & 1.4 & $\csname FourierN1.4K\endcsname$ \\\hline\hline
    \end{tabular}
    \label{tab:ns}
\end{table}



