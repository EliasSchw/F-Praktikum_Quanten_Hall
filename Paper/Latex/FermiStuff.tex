\section{Fermi wave vector, Fermi energy and Fermi velocity}
To determine all Fermi relatied values assuming a parabolic energy dispersion, firstly the Fermi wave vector $k_\text{F}$ is calculated with eq\,\ref{eq:kFermi}.
\begin{align}
    k_\text{F}=\sqrt{2\pi n_\text{slope}} \label{eq:kFermi}
\end{align}
After that the Fermi energy $E_\text{F}$ can be calculated with the obtained wave vector and the cyclotron mass as given in eq.\,\ref{eq:EFermi}.
\begin{align}
    E_\text{F}=\frac{\hbar^2k_\text{F}^2}{2m_\text{c}} \label{eq:EFermi}
\end{align}
Lastly, the Fermi velocity $v_\text{F}$ can be calculated with eq.\,\ref{eq:vFermi}.
\begin{align}
    v_\text{F}=\frac{\hbar k_\text{F}}{m_\text{c}} \label{eq:vFermi}
\end{align}
The results are shown in tab.\,\ref{tab:FermiValues}.
\begin{table}[h]
    \centering
    \begin{tabular}{c|c|c|c}
        \hline\hline
        $T\,\text{K}$ & $k_\text{F}\cdot10^{8}\,\text{m}^{-1}$ & $E_\text{F}\cdot10^{-2}\,\text{eV}$ & $v_\text{F}\cdot10^{5}\,\text{m/s}$ \\\hline
        $1.4$ & $\csname kFermi1.4K\endcsname$ & $\csname EFermi1.4K\endcsname$ & $\csname vFermi1.4K\endcsname$ \\
        $2.1$ & $\csname kFermi2.1K\endcsname$ & $\csname EFermi2.1K\endcsname$ & $\csname vFermi2.1K\endcsname$ \\
        $3$ & $\csname kFermi3K\endcsname$ & $\csname EFermi3K\endcsname$ & $\csname vFermi3K\endcsname$ \\
        $4.2$ & $\csname kFermi4.2K\endcsname$ & $\csname EFermi4.2K\endcsname$ & $\csname vFermi4.2K\endcsname$ \\
        \hline
        \hline
        
    \end{tabular}
\end{table}
The errors are calculated using gaussian error propagation. For the cyclotron mass no temperature dependence is expected.
Therefore, the only temperature dependent value is the charge carrier density. The temperature dependence of $n_\text{slope}$
can be seen in the results for the Fermi wave vector. Those values do not agree with themself within their errors and show
a slight dependence for higher temperatures towards higher values. The $k_\text{F}$ value for $2.1\,\text{K}$ is in this case
treated as an outlier. For the Fermi energy and the Fermi velocity, the temperature dependence is barely visible, as 
the impact of $n_\text{slope}$ is much smaller. \\
It's interesting to note that the Fermi Energy is well below the thermal energy at room temperature,
which is about $25\,\text{meV}$. In the context of the experiment this means that the confinement in stacking direction
of our quantum well would not be strong enough to keep the electrons in the ground state. That means that it's
not possible to perform the experiment at room temperature because the system would not longer be 2D.
To neglect those thermal effect, the experiment is performed at lower temperatures.