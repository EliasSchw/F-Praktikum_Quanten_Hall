\section{Electronic properties at zero magnetic field}
In this section, the electronic properties of the sample at zero magnetic field are described.
The dependence between longitudinal resistivity $\rho_\text{xx}$ and the longitudinal conductivity $\sigma_\text{xx}$
for $B=0$ is shown in eq.\,\ref{eq:simaxx}.
\begin{align}
    \sigma_{xx} = \frac{1}{\rho_{xx}}
    \label{eq:simaxx}
\end{align}
In this case, $\rho_\text{xx}$ and $\sigma_\text{xx}$ are treated as scalars, because the off diagonal elements of the resistivity vanish.
For magnetic fields higher than zero, the off diagonal elements not longer vanish and therefore the two properties are needed to be treated as
tensors. The dependency is given by eq.\,\ref{eq:sigmaxxtensor}
\begin{align}
    \hat{\sigma}_\text{xx} = \frac{\hat{\rho}_\text{xx}}{\hat{\rho}_\text{xx}^2 + \hat{\rho}_\text{xy}^2}
    \label{eq:sigmaxxtensor}
\end{align}
With the Hall factor $R_\text{H}$, which is defined as shown in eq.\,\ref{eq:RH}
\begin{align}
    R_\text{H}=\frac{1}{en}
\end{align}
the charge carrier mobility $\mu$ can be calculated using eq.\,\ref{eq:mu}.
\begin{align}
    \mu = R_\text{H} \cdot \sigma_{xx}
    \label{eq:mu}
\end{align}
The results are shown in tab.\,\ref{tab:zerofield}.
\begin{table}[h]
    \centering
    \begin{tabular}{c|c}
        \hline\hline
        $T / \text{K}$ & $\sigma_\text{xx}\,/\,10^{-3}\,\Omega^{-1}$\\\hline\hline
        4.2 & $\csname simaXX4.2K\endcsname$ \\
        3 & $\csname simaXX3K\endcsname$ \\
        2.1 & $\csname simaXX2.1K\endcsname$ \\
        1.4 & $\csname simaXX1.4K\endcsname$ \\\hline\hline
        $T / \text{K}$ & $R_\text{H}\,/\,10^{3}\,m^2C^{-1}$\\\hline\hline
        4.2 & $\csname RHall4.2K\endcsname$ \\
        3 & $\csname RHall3K\endcsname$ \\
        2.1 & $\csname RHall2.1K\endcsname$ \\
        1.4 & $\csname RHall1.4K\endcsname$ \\\hline\hline
        $T / \text{K}$ & $\mu\,/\,10^{1}\,m^2V^{-1}s^{-1}$\\\hline\hline
        4.2 & $\csname my4.2K\endcsname$ \\
        3 & $\csname my3K\endcsname$ \\
        2.1 & $\csname my2.1K\endcsname$ \\
        1.4 & $\csname my1.4K\endcsname$ \\\hline\hline
    \end{tabular}
\end{table}
The error of $\sigma_\text{xx}$ is given by gaussian error propagation of the relative error of $\rho_\text{xx}$ with $\sqrt{2}\,\%$ (see ch.\,\ref{sec:setup}).
For the caluclation of $R_\text{H}$, $n_\text{slope}$ is used, because this method is the most precise one.
The error of $R_\text{H}$ and $\mu$ is then also calculated by gaussian error propagation.
Looking at the results, no temperature dependency is visible for all three properties. For $\sigma_\text{xx}$ and
$\mu$ within their error, the values are constant. The Hall factor $R_\text{H}$ also seems to be constant for the different temperatures,
however the data does not agree with itself within the error. This is due to the fact, that the error is very small 
and yet the values are very close to each other. 
