\section{Electronic properties at zero magnetic field}
This section describes the electronic properties of the sample at zero magnetic field.
The dependence between the resistivity $\rho$ and the conductivity $\sigma$ for $B=0$ in an isotropic material is given by:
\begin{align}
    \sigma = \frac{1}{\rho}.
    \label{eq:simaxx}
\end{align}
In this case, $\rho$ and $\sigma$ are treated as scalars because the off-diagonal elements of the resistivity vanish.
For non-zero magnetic fields, the off-diagonal elements no longer vanish, so $\rho$ and $\sigma$ must be treated as tensors. 
The dependence is given by:
\begin{align}
    \hat{\sigma}_\text{xx} = \frac{\hat{\rho}_\text{xx}}{\hat{\rho}_\text{xx}^2 + \hat{\rho}_\text{xy}^2}.
    \label{eq:sigmaxxtensor}
\end{align}
With the Hall factor $R_\text{H}$
\begin{align}
    R_\text{H}=\frac{1}{en}\label{eq:RH}
\end{align}
the charge carrier mobility $\mu$ can be calculated using 
\begin{align}
    \mu = R_\text{H} \cdot \sigma_{xx}
    \label{eq:mu}
\end{align}
The results are shown in tab.\,\ref{tab:zerofield}.
\begin{table}[h]
    \centering
    \begin{tabular}{c|c}
        \hline\hline
        $T / \text{K}$ & $\sigma_\text{xx}\,/\,10^{-3}\,\Omega^{-1}$\\\hline
        4.2 & $\csname simaXX4.2K\endcsname$ \\
        3 & $\csname simaXX3K\endcsname$ \\
        2.1 & $\csname simaXX2.1K\endcsname$ \\
        1.4 & $\csname simaXX1.4K\endcsname$ \\\hline\hline
        $T / \text{K}$ & $R_\text{H}\,/\,10^{3}\,\text{m}^2\text{C}^{-1}$\\\hline
        4.2 & $\csname RHall4.2K\endcsname$ \\
        3 & $\csname RHall3K\endcsname$ \\
        2.1 & $\csname RHall2.1K\endcsname$ \\
        1.4 & $\csname RHall1.4K\endcsname$ \\\hline\hline
        $T / \text{K}$ & $\mu\,/\,10^{1}\,\text{m}^2\text{V}^{-1}\text{s}^{-1}$\\\hline
        4.2 & $\csname my4.2K\endcsname$ \\
        3 & $\csname my3K\endcsname$ \\
        2.1 & $\csname my2.1K\endcsname$ \\
        1.4 & $\csname my1.4K\endcsname$ \\\hline\hline
    \end{tabular}
    \caption{Values of the longitudinal conductivity, Hall constant and mobility at zero magnetic field \label{tab:zerofield}}
\end{table}
The error of $\sigma_\text{xx}$ is given by Gaussian error propagation of the relative error of $\rho_\text{xx}$ with $1.4\,\%$.
For the calculation of $R_\text{H}$, $n_\text{slope}$ is used, because this method is the most precise.
The error of $R_\text{H}$ and $\mu$ is then also calculated by Gaussian error propagation.
Looking at the results, there is no temperature dependence for any of the three properties. 
Taking the errors into account, no temperature dependence can be observed. 
The Hall factor $R_\text{H}$ also seems to be constant for the different temperatures, 
but the data do not agree with each other within the error. 
This is due to the fact that the error of the linear regression is very small and yet the values are very close.
