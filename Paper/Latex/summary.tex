\section{Summary}

In this work, the Quantum Hall Effect (QHE) was investigated in a two-dimensional electron system 
at low temperatures and high magnetic fields. 
The QHE provides a precise method to measure the ratio of Planck's constant $h$ to the squared elementary charge $e^2$,
leading to the determination of the von Klitzing constant $R_\text{K}$. Our measurements confirm the 
quantization of the Hall resistivity $\rho_\text{xy}$ in multiples of $h/e^2$, with the experimentally 
determined $R_\text{K}$ values agreeing with the exact value of $R_\text{K, exact} = \csname Klitzing1.4K\endcsname\,\text{k}\Omega$.
The charge carrier density $n$ was determined using three different methods: 
the classical Hall effect, the quantum Hall plateaus, and the Shubnikov-de Haas oscillations. 
The results were consistent across methods, with $n$ values in the range of $\csname nEins1.4K\endcsname \cdot 10^{15}\,\text{m}^{-2}$, 
showing a slight temperature dependence.
The cyclotron mass $m_\text{c}$ was calculated from the temperature dependence 
of the oscillation amplitudes in the longitudinal conductivity $\sigma_\text{xx}$. 
The determined values, $m_\text{c} = (\csname cyclotron15\endcsname)\cdot 10^{-2}m_\text{e}$ for the $1.4\,\text{K}$ and $3\,\text{K}$ 
pair and $m_\text{c} = (\csname cyclotron21\endcsname)\cdot 10^{-2}m_\text{e}$ for the $2.1\,\text{K}$ and $4.2\,\text{K}$ pair, 
are in good agreement with literature values.
Additionally, the Fermi wave vector $k_\text{F}$, Fermi energy $E_\text{F}$, and Fermi velocity $v_\text{F}$ 
were calculated assuming a parabolic energy dispersion. For $T = 1.4\,\text{K}$, 
the results were $k_\text{F} = (\csname kFermi1.4K\endcsname) \cdot 10^{8}\,\text{m}^{-1}$, $E_\text{F} = (\csname EFermi1.4K\endcsname)\cdot 10^{-2}\,\text{eV}$, 
and $v_\text{F} = (\csname vFermi1.4K\endcsname) \cdot 10^{5}\,\text{m/s}$. 
These values highlight the necessity of low temperatures to observe the QHE.
No evidence of the fractional Quantum Hall Effect (FQHE) was observed in the accessible magnetic field range. 
The absence of fractional plateaus and corresponding longitudinal resistivity peaks suggests that higher magnetic 
fields are required to observe this phenomenon.
Effects of spin orbit coupling could be observed.
The experimental results confirm the fundamental properties of the Quantum Hall Effect 
and provide insights into the electronic properties of the system. The findings are consistent with literature values and 
highlight the precision of the QHE as a tool for fundamental physics and metrology.
