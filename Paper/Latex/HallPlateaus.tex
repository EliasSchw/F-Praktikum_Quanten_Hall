\section{Von Klitzing Constant}

The Hall resistivity $\rho_{\text{xy}}$ forms plateaus at
\begin{align}
    \rho_{xy} = \frac{h}{\nu e^2} =: \frac{R_\text K}{\nu}
\end{align}
with the elementary charge $e$, the planc constant $h$. 
The filling factor $\nu \in \mathbb N$ describes the number of occupied Landau levels.

The goal of this section is to determine the von Klitzing constant $R_K$.
Before reading off the plateaus, two errors have to be taken into account:
the time delay and the different scales of the lock in amplifiers.
To account for the time delay, 
caused by the integration and perhhaps some additional internal delay of the lock in amplifiers,
the three voltages of the lock in amplifiers are shifted backwards in time.
Since the magnetic field B is ramped up and down between $0T$ and $9T$, 
the constant shift in time can be chosen, s.t. both graphs match.
To compensate for different scales $\alpha_i$, 
the lock in amplifiers $1$ and $2$ measuring $U_xy$ and $U_I$ are exchanged and the measurement at $1.4K$ is repeated.
Using
\begin{align}
    R_{\text K,1} = \frac{U_{\text{xy}, 1}}{I_2} = \frac{\alpha_1 U}{\alpha_2 I}\\
    R_{\text K,2} = \frac{U_{\text{xy}, 2}}{I_1} = \frac{\alpha_2 U}{\alpha_1 I}
\end{align}
one finds the correction factor
\begin{align}
    \alpha = \frac{\alpha_1}{\alpha_2} = \sqrt{\frac{U_{\text{xy},1}I_2}{U_{\text{xy}, 2}I_1}}. 
\end{align}
All further calculations are corrected for the time delay and the different scales.
Only $\rho_{\text{xy}}$ is corrected with the factor $\alpha$, 
since no additional measurement with exchanged lock in amplifiers was performed.
In principle, $\rho_{\text{xx}}$ could be corrected in the same manner.
\\
To obtain $R_\text K = \rho_{\text{xy}}$, the average value of the plateaus is taken. 
The $R_\text K$ are presented in tab. \ref{tab:plateau_values}.
\begin{table}[h!]
    \centering
    \begin{tabular}{c|c c c c}
        $\nu$  & $4.2\,\text{K}$        & $3.0\,\text{K}$        & $2.1\,\text{K}$        & $1.4\,\text{K}$        \\ \hline
        1      & $\csname PlateauNr14.2K\endcsname$  & $\csname PlateauNr13K\endcsname$  & $\csname PlateauNr12.1K\endcsname$  & $\csname PlateauNr11.4K\endcsname$  \\ 
        2      & $\csname PlateauNr24.2K\endcsname$  & $\csname PlateauNr23K\endcsname$  & $\csname PlateauNr22.1K\endcsname$  & $\csname PlateauNr21.4K\endcsname$  \\ 
        3      & $\csname PlateauNr34.2K\endcsname$  & $\csname PlateauNr33K\endcsname$  & $\csname PlateauNr32.1K\endcsname$  & $\csname PlateauNr31.4K\endcsname$  \\ 
        4      & $\csname PlateauNr44.2K\endcsname$  & $\csname PlateauNr43K\endcsname$  & $\csname PlateauNr42.1K\endcsname$  & $\csname PlateauNr41.4K\endcsname$  \\ 
        5      & $\csname PlateauNr54.2K\endcsname$  & $\csname PlateauNr53K\endcsname$  & $\csname PlateauNr52.1K\endcsname$  & $\csname PlateauNr51.4K\endcsname$  \\ 
    \end{tabular}
    \caption{$R_\text K$ for different temperatures and filling factors in $10^{4}\,\Omega$.}
    \label{tab:plateau_values}
\end{table}







