\section{Introduction}
The Quantum Hall Effect (QHE) is one of the simplest ways to measure the ratio between Planck's constant 
and the elementary charge. In 1985 Klaus von Klitzing discovered the QHE in a two-dimensional system which
desribes the quantization of the transverse resistivity $\rho_\text{xy}$ in multiples of $h/e^2$ \cite{Nobelpreis}, later called the von 
Klitzing constant.
This made it possible to define a new resistivity standard that depends only on fundamental constants. 
In the following sections, the QHE and it's related properties are determined and discussed. 
The main part of this paper is the analysis of the transverse resistivity and the appearance of the Hall plateaus.
The carrier density is determined by different methods and compared. 
The cyclotron mass, Fermi energy and Fermi velocity are also calculated.
Finally, we will investigate whether a spin-orbit coupling effect can be observed 
in the system and search for plateaus of the fractional quantum Hall effect.\\

